\section{Example Workflow}

\begin{frame}[fragile]{Developer Workflow Example - Part 1a: Capture}
  \begin{block}{Scenario: Building Authentication Service in Python}
    \begin{enumerate}
      \item \textbf{Initialize capture session}
      \begin{lstlisting}[basicstyle=\ttfamily\scriptsize]
$ osp capture --session auth-service \
  --ai-tool claude-code \
  --tags "authentication,jwt,fastapi"
Session started: auth-service-20251009-103000
Monitoring AI interactions...
      \end{lstlisting}

      \item \textbf{First prompt iteration}
      \begin{itemize}
        \item Prompt: "Implement JWT authentication in Python using FastAPI"
        \item Captured context:
          \begin{itemize}
            \item Current directory structure
            \item Existing dependencies in requirements.txt
            \item Python version (3.11)
          \end{itemize}
        \item Files generated: \texttt{src/auth.py}, \texttt{tests/test\_auth.py}
        \item Metadata: AI model version, timestamp, token count
      \end{itemize}
    \end{enumerate}
  \end{block}
\end{frame}

\begin{frame}[fragile]{Developer Workflow Example - Part 1b: Capture}
  \begin{block}{Scenario: Building Authentication Service in Python}
    \begin{enumerate}
      \setcounter{enumi}{2}
      \item \textbf{Second prompt iteration}
      \begin{itemize}
        \item Prompt: "Add refresh token support with Redis caching"
        \item Captured context:
          \begin{itemize}
            \item Previous prompt (id: 1) as dependency
            \item Redis configuration in docker-compose.yml
            \item Security requirement: token rotation
          \end{itemize}
        \item Files modified: \texttt{src/auth.py}
        \item Files generated: \texttt{src/refresh\_tokens.py}, \texttt{src/redis\_client.py}
      \end{itemize}
    \end{enumerate}
  \end{block}
\end{frame}

\begin{frame}[fragile]{Developer Workflow Example - Part 2a: Commit}
  \begin{block}{Committing Code with Prompts}
    \begin{lstlisting}[basicstyle=\ttfamily\scriptsize]
$ git add src/ tests/ .prompts/
$ git commit -m "Add JWT authentication with refresh tokens"
[pre-commit hook: validating .prompts/auth-service.yaml]
Schema validation: PASS
All required fields present: PASS
Git hash linked: a1b2c3d4e5f6
[main a1b2c3d4] Add JWT authentication with refresh tokens
 5 files changed, 342 insertions(+)
 create mode 100644 .prompts/auth-service.yaml
 create mode 100644 src/auth.py
 create mode 100644 src/refresh_tokens.py
 create mode 100644 src/redis_client.py
 create mode 100644 tests/test_auth.py
    \end{lstlisting}
  \end{block}
\end{frame}

\begin{frame}[fragile]{Developer Workflow Example - Part 2b: Commit}
  \begin{block}{Generated Prompt File Structure}
    \begin{itemize}
      \item \texttt{.prompts/auth-service.yaml} includes:
        \begin{itemize}
          \item Complete prompt history (2 iterations)
          \item All context and dependencies
          \item File tracking and git linkage
          \item Validation results (tests passed, coverage 89\%)
        \end{itemize}
    \end{itemize}
  \end{block}
\end{frame}

\begin{frame}[fragile]{Developer Workflow Example - Part 3: Replay}
  \begin{block}{Cross-Language Porting to Rust}
    \begin{lstlisting}[basicstyle=\ttfamily\scriptsize]
$ osp replay --input .prompts/auth-service.yaml \
  --target-lang rust --framework actix-web \
  --output .prompts/auth-service-rust.yaml
Loading prompt history... 2 prompts found
Adapting context for Rust environment...
  Dependencies: fastapi -> actix-web,
                pyjwt -> jsonwebtoken
  Async model: asyncio -> tokio
  Redis client: redis-py -> redis-rs
    \end{lstlisting}
  \end{block}
\end{frame}

\begin{frame}[fragile]{Developer Workflow Example - Part 3b: Replay Results}
  \begin{block}{Execution Results}
    \begin{lstlisting}[basicstyle=\ttfamily\scriptsize]
Executing adapted prompt 1...
  Generated: src/auth.rs, tests/auth_test.rs

Executing adapted prompt 2...
  Generated: src/refresh_tokens.rs
  Generated: src/redis_client.rs
  Modified: src/auth.rs

Validation:
  Cargo build: SUCCESS
  Cargo test: 8/8 passed
  Coverage: 87%

Prompt history saved to:
  .prompts/auth-service-rust.yaml
    \end{lstlisting}
  \end{block}
\end{frame}

\begin{frame}{Workflow Benefits - Technical Detail (Part 1)}
  \begin{block}{Reproducibility}
    \begin{itemize}
      \item Exact prompt sequence preserved
      \item Environment context captured (versions, dependencies)
      \item Validation metrics tracked (tests, coverage)
      \item Git linkage enables bisecting prompt history
    \end{itemize}
  \end{block}

  \begin{block}{Cross-Language Adaptation}
    \begin{itemize}
      \item Framework mapping: FastAPI $\rightarrow$ Actix-web
      \item Library translation: pyjwt $\rightarrow$ jsonwebtoken
      \item Async model adaptation: asyncio $\rightarrow$ tokio
      \item Maintains architectural intent across languages
    \end{itemize}
  \end{block}
\end{frame}

\begin{frame}{Workflow Benefits - Technical Detail (Part 2)}
  \begin{block}{Knowledge Sharing}
    \begin{itemize}
      \item Searchable by tags: "authentication", "jwt", "fastapi"
      \item Discoverable by file patterns: "auth*.py"
      \item Linkable in documentation and PRs
      \item Reusable for onboarding and training
    \end{itemize}
  \end{block}
\end{frame}
